\documentclass[]{article}
\usepackage{lmodern}
\usepackage{amssymb,amsmath}
\usepackage{ifxetex,ifluatex}
\usepackage{fixltx2e} % provides \textsubscript
\ifnum 0\ifxetex 1\fi\ifluatex 1\fi=0 % if pdftex
  \usepackage[T1]{fontenc}
  \usepackage[utf8]{inputenc}
\else % if luatex or xelatex
  \ifxetex
    \usepackage{mathspec}
    \usepackage{xltxtra,xunicode}
  \else
    \usepackage{fontspec}
  \fi
  \defaultfontfeatures{Mapping=tex-text,Scale=MatchLowercase}
  \newcommand{\euro}{€}
\fi
% use upquote if available, for straight quotes in verbatim environments
\IfFileExists{upquote.sty}{\usepackage{upquote}}{}
% use microtype if available
\IfFileExists{microtype.sty}{%
\usepackage{microtype}
\UseMicrotypeSet[protrusion]{basicmath} % disable protrusion for tt fonts
}{}
\usepackage[margin=1in]{geometry}
\usepackage{color}
\usepackage{fancyvrb}
\newcommand{\VerbBar}{|}
\newcommand{\VERB}{\Verb[commandchars=\\\{\}]}
\DefineVerbatimEnvironment{Highlighting}{Verbatim}{commandchars=\\\{\}}
% Add ',fontsize=\small' for more characters per line
\usepackage{framed}
\definecolor{shadecolor}{RGB}{248,248,248}
\newenvironment{Shaded}{\begin{snugshade}}{\end{snugshade}}
\newcommand{\KeywordTok}[1]{\textcolor[rgb]{0.13,0.29,0.53}{\textbf{{#1}}}}
\newcommand{\DataTypeTok}[1]{\textcolor[rgb]{0.13,0.29,0.53}{{#1}}}
\newcommand{\DecValTok}[1]{\textcolor[rgb]{0.00,0.00,0.81}{{#1}}}
\newcommand{\BaseNTok}[1]{\textcolor[rgb]{0.00,0.00,0.81}{{#1}}}
\newcommand{\FloatTok}[1]{\textcolor[rgb]{0.00,0.00,0.81}{{#1}}}
\newcommand{\CharTok}[1]{\textcolor[rgb]{0.31,0.60,0.02}{{#1}}}
\newcommand{\StringTok}[1]{\textcolor[rgb]{0.31,0.60,0.02}{{#1}}}
\newcommand{\CommentTok}[1]{\textcolor[rgb]{0.56,0.35,0.01}{\textit{{#1}}}}
\newcommand{\OtherTok}[1]{\textcolor[rgb]{0.56,0.35,0.01}{{#1}}}
\newcommand{\AlertTok}[1]{\textcolor[rgb]{0.94,0.16,0.16}{{#1}}}
\newcommand{\FunctionTok}[1]{\textcolor[rgb]{0.00,0.00,0.00}{{#1}}}
\newcommand{\RegionMarkerTok}[1]{{#1}}
\newcommand{\ErrorTok}[1]{\textbf{{#1}}}
\newcommand{\NormalTok}[1]{{#1}}
\usepackage{graphicx}
\makeatletter
\def\maxwidth{\ifdim\Gin@nat@width>\linewidth\linewidth\else\Gin@nat@width\fi}
\def\maxheight{\ifdim\Gin@nat@height>\textheight\textheight\else\Gin@nat@height\fi}
\makeatother
% Scale images if necessary, so that they will not overflow the page
% margins by default, and it is still possible to overwrite the defaults
% using explicit options in \includegraphics[width, height, ...]{}
\setkeys{Gin}{width=\maxwidth,height=\maxheight,keepaspectratio}
\ifxetex
  \usepackage[setpagesize=false, % page size defined by xetex
              unicode=false, % unicode breaks when used with xetex
              xetex]{hyperref}
\else
  \usepackage[unicode=true]{hyperref}
\fi
\hypersetup{breaklinks=true,
            bookmarks=true,
            pdfauthor={Lorenz Gerber},
            pdftitle={Data analysis protocol for OU3},
            colorlinks=true,
            citecolor=blue,
            urlcolor=blue,
            linkcolor=magenta,
            pdfborder={0 0 0}}
\urlstyle{same}  % don't use monospace font for urls
\setlength{\parindent}{0pt}
\setlength{\parskip}{6pt plus 2pt minus 1pt}
\setlength{\emergencystretch}{3em}  % prevent overfull lines
\setcounter{secnumdepth}{0}

%%% Use protect on footnotes to avoid problems with footnotes in titles
\let\rmarkdownfootnote\footnote%
\def\footnote{\protect\rmarkdownfootnote}

%%% Change title format to be more compact
\usepackage{titling}

% Create subtitle command for use in maketitle
\newcommand{\subtitle}[1]{
  \posttitle{
    \begin{center}\large#1\end{center}
    }
}

\setlength{\droptitle}{-2em}
  \title{Data analysis protocol for OU3}
  \pretitle{\vspace{\droptitle}\centering\huge}
  \posttitle{\par}
  \author{Lorenz Gerber}
  \preauthor{\centering\large\emph}
  \postauthor{\par}
  \predate{\centering\large\emph}
  \postdate{\par}
  \date{13 December 2015}



\begin{document}

\maketitle


This document describes the data processing and analysis for the
mandatory assignment `OU3'. First some names and definitions are set:

\begin{Shaded}
\begin{Highlighting}[]
\CommentTok{# setting the workdirectory}
\KeywordTok{setwd}\NormalTok{(}\StringTok{'~/github/table_analysis/'}\NormalTok{)}

\CommentTok{# names of the data types tested}
\NormalTok{datatype<-}\KeywordTok{c}\NormalTok{(}\StringTok{'array'}\NormalTok{, }\StringTok{'dlist'}\NormalTok{, }\StringTok{'mtf'}\NormalTok{)}

\CommentTok{# number of operations in each test case}
\NormalTok{datapoints<-}\KeywordTok{c}\NormalTok{(}\DecValTok{250}\NormalTok{, }\DecValTok{500}\NormalTok{, }\DecValTok{1000}\NormalTok{, }\DecValTok{2500}\NormalTok{, }\DecValTok{5000}\NormalTok{, }\DecValTok{7500}\NormalTok{, }\DecValTok{10000}\NormalTok{)}

\CommentTok{# reading the raw data filenames into variables}
\NormalTok{array<-}\KeywordTok{list.files}\NormalTok{(}\DataTypeTok{pattern =} \StringTok{'^a.'}\NormalTok{)}
\NormalTok{dlist<-}\KeywordTok{list.files}\NormalTok{(}\DataTypeTok{pattern =} \StringTok{'^d.'}\NormalTok{)}
\NormalTok{mtf<-}\KeywordTok{list.files}\NormalTok{(}\DataTypeTok{pattern =} \StringTok{'^m.'}\NormalTok{)}

\CommentTok{# creating result matrix}
\NormalTok{results<-}\KeywordTok{matrix}\NormalTok{(}\DecValTok{105}\NormalTok{, }\DecValTok{7}\NormalTok{, }\DecValTok{15}\NormalTok{)}

\CommentTok{# value insertion sequence into results matrix}
\NormalTok{col_ind<-}\KeywordTok{seq}\NormalTok{(}\DecValTok{1}\NormalTok{,}\DecValTok{15}\NormalTok{,}\DecValTok{3}\NormalTok{)}

\CommentTok{# row and column names for result tabme}
\KeywordTok{colnames}\NormalTok{(results)<-}\KeywordTok{rep}\NormalTok{(}\KeywordTok{c}\NormalTok{(}\StringTok{'array'}\NormalTok{, }\StringTok{'dlist'}\NormalTok{, }\StringTok{'mtf'}\NormalTok{),}\DecValTok{5}\NormalTok{)}
\KeywordTok{rownames}\NormalTok{(results)<-}\KeywordTok{c}\NormalTok{(}\DecValTok{250}\NormalTok{, }\DecValTok{500}\NormalTok{, }\DecValTok{1000}\NormalTok{, }\DecValTok{2500}\NormalTok{, }\DecValTok{5000}\NormalTok{, }\DecValTok{7500}\NormalTok{, }\DecValTok{10000}\NormalTok{)}

\CommentTok{# names of conducted experiments}
\NormalTok{experiments<-}\KeywordTok{c}\NormalTok{(}\StringTok{'Insertion'}\NormalTok{, }\StringTok{'RandomExistingLookup'}\NormalTok{, }
               \StringTok{'RandomNonExisitingLookup'}\NormalTok{, }\StringTok{'SkewedLookup'}\NormalTok{, }\StringTok{'Remove'}\NormalTok{)}
\end{Highlighting}
\end{Shaded}

Then the mean and sd for each datatype, datapoint and test is
calculated:

\begin{Shaded}
\begin{Highlighting}[]
\CommentTok{# looping through datatypes}
\NormalTok{for (dt_i in }\DecValTok{1}\NormalTok{:}\DecValTok{3} \NormalTok{)\{}
    \CommentTok{# looping through datapoints}
    \NormalTok{for ( exp_i in }\DecValTok{1}\NormalTok{:}\DecValTok{7} \NormalTok{)\{}
        \CommentTok{# dynamically select file to process}
        \NormalTok{current<-}\KeywordTok{eval}\NormalTok{(}\KeywordTok{parse}\NormalTok{(}\DataTypeTok{text=}\KeywordTok{paste}\NormalTok{(datatype[dt_i], }\StringTok{'['}\NormalTok{,exp_i,}\StringTok{']'}\NormalTok{)))}
        \CommentTok{# load current file}
        \NormalTok{current_data<-}\KeywordTok{read.table}\NormalTok{(}\DataTypeTok{file=}\NormalTok{current, }\DataTypeTok{sep=}\StringTok{'}\CharTok{\textbackslash{}t}\StringTok{'}\NormalTok{, }\DataTypeTok{dec=}\StringTok{'.'}\NormalTok{)}
        \CommentTok{# calculate statistics and write into result table}
        \NormalTok{results[exp_i, col_ind+dt_i}\DecValTok{-1}\NormalTok{]<-}\KeywordTok{apply}\NormalTok{(current_data,}\DecValTok{2}\NormalTok{, mean)}
    \NormalTok{\}}
\NormalTok{\}}

\NormalTok{means<-results}
\end{Highlighting}
\end{Shaded}

Calculating Standard Deviations and Relative Standard Deviation (RSD)

\begin{Shaded}
\begin{Highlighting}[]
\CommentTok{# looping through datatypes}
\NormalTok{for (dt_i in }\DecValTok{1}\NormalTok{:}\DecValTok{3} \NormalTok{)\{}
    \CommentTok{# looping through datapoints}
    \NormalTok{for ( exp_i in }\DecValTok{1}\NormalTok{:}\DecValTok{7} \NormalTok{)\{}
        \CommentTok{# dynamically select file to process}
        \NormalTok{current<-}\KeywordTok{eval}\NormalTok{(}\KeywordTok{parse}\NormalTok{(}\DataTypeTok{text=}\KeywordTok{paste}\NormalTok{(datatype[dt_i], }\StringTok{'['}\NormalTok{,exp_i,}\StringTok{']'}\NormalTok{)))}
        \CommentTok{# load current file}
        \NormalTok{current_data<-}\KeywordTok{read.table}\NormalTok{(}\DataTypeTok{file=}\NormalTok{current, }\DataTypeTok{sep=}\StringTok{'}\CharTok{\textbackslash{}t}\StringTok{'}\NormalTok{, }\DataTypeTok{dec=}\StringTok{'.'}\NormalTok{)}
        \CommentTok{# calculate statistics and write into result table}
        \NormalTok{results[exp_i, col_ind+dt_i}\DecValTok{-1}\NormalTok{]<-}\KeywordTok{apply}\NormalTok{(current_data,}\DecValTok{2}\NormalTok{, sd)}
    \NormalTok{\}}
\NormalTok{\}}

\NormalTok{sds<-results}

\CommentTok{# calculating relative standard deviation}
\NormalTok{rsds<-}\DecValTok{100}\NormalTok{/means*sds}
\end{Highlighting}
\end{Shaded}

Plotting the `insertion' experiment. Plot (a) shows the speed in
microseconds and plot (b) the RSDs.
\includegraphics{data_analysis_files/figure-latex/unnamed-chunk-4-1.pdf}

Plotting the `RandomExistingLookup' experiment. Plot (a) shows the speed
in microseconds and plot (b) the RSDs.
\includegraphics{data_analysis_files/figure-latex/unnamed-chunk-5-1.pdf}

Plotting the `RandomNonExistingLookup' experiment. Plot (a) shows the
speed in microseconds and plot (b) the RSDs.
\includegraphics{data_analysis_files/figure-latex/unnamed-chunk-6-1.pdf}

Plotting the `SkewedLookup' experiment. Plot (a) shows the speed in
microseconds and plot (b) the RSDs.
\includegraphics{data_analysis_files/figure-latex/unnamed-chunk-7-1.pdf}

Plotting the `remove' experiment. Plot (a) shows the speed in
microseconds and plot (b) the RSDs.
\includegraphics{data_analysis_files/figure-latex/unnamed-chunk-8-1.pdf}

Plotting the several experiments of dlist datatype. Plot (a) shows the
speed in microseconds and plot (b) the RSDs.
\includegraphics{data_analysis_files/figure-latex/unnamed-chunk-9-1.pdf}

Plotting the several experiments of mtf datatype. Plot (a) shows the
speed in microseconds and plot (b) the RSDs.
\includegraphics{data_analysis_files/figure-latex/unnamed-chunk-10-1.pdf}

Plotting the several experiments of array datatype. Plot (a) shows the
speed in microseconds and plot (b) the RSDs.
\includegraphics{data_analysis_files/figure-latex/unnamed-chunk-11-1.pdf}

\end{document}
